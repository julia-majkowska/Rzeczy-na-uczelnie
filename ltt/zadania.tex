% Copyright 2004 by Till Tantau <tantau@users.sourceforge.net>.
%
% In principle, this file can be redistributed and/or modified under
% the terms of the GNU Public License, version 2.
%
% However, this file is supposed to be a template to be modified
% for your own needs. For this reason, if you use this file as a
% template and not specifically distribute it as part of a another
% package/program, I grant the extra permission to freely copy and
% modify this file as you see fit and even to delete this copyright
% notice. 

\documentclass{article}
\usepackage{polski}
\usepackage[utf8]{inputenc}
\usepackage{eucal}

% There are many different themes available for Beamer. A comprehensive
% list with examples is given here:
% http://deic.uab.es/~iblanes/beamer_gallery/index_by_theme.html
% You can uncomment the themes below if you would like to use a different
% one:
%\usetheme{AnnArbor}
%\usetheme{Antibes}
%\usetheme{Bergen}
%\usetheme{Berkeley}
%\usetheme{Berlin}
%\usetheme{Boadilla}
%\usetheme{boxes}
%\usetheme{CambridgeUS}
%\usetheme{Copenhagen}
%\usetheme{Darmstadt}
%\usetheme{default}
%\usetheme{Frankfurt}
%\usetheme{Goettingen}
%\usetheme{Hannover}
%\usetheme{Ilmenau}
%\usetheme{JuanLesPins}
%\usetheme{Luebeck}
%\usetheme{Madrid}
%\usetheme{Malmoe}
%\usetheme{Marburg}
%\usetheme{Montpellier}
%\usetheme{PaloAlto}
%\usetheme{Pittsburgh}
%\usetheme{Rochester}
%\usetheme{Singapore}
%\usetheme{Szeged}
\begin{document}

\section{Ćwiczenia}

\begin{enumerate}

	\item Udowodnij w systemie dedukcji naturalnej używając termów logiki intuicjonistycznej. Potaraj się, aby powstały dowód był znormalizowany.
	\begin{enumerate}
 \item \( \bot \rightarrow p\)
 \item \( p \rightarrow \neg \neg p \)
 \item \( \neg \neg \neg p \rightarrow \neg p\)
 \item \( (p \rightarrow q) \rightarrow (\neg q \rightarrow \neg p ) \)
 \item \( (\neg p \vee \neg q) \rightarrow \neg ( q \wedge p ) \)
 \item \( (( p \wedge q) \rightarrow r) \rightarrow ( p \rightarrow ( q \rightarrow r )) \)
\end{enumerate}
	\item Znajdź lambda termy odpowiadające tym dowodom. 
	\item Udowodnij,  że \( a \leq b \Leftrightarrow a \cup b = b\) zdefiniowana na algebrze boolowskiej jest porządkiem częściowym oraz: 
	\begin{enumerate}
	\item \( a \cap b  \leq a\)
	\item \( a \leq b\) wtw \( a \cap b = a\)
	\item \( \cap , \cup\) to odpowiednio infimum i supremum względem porządku \( \leq\) 
	\item  0 i 1 to odpowiednio najmiejszy i największy element w tym porządku.
	\end{enumerate}
	\item Udowodnij izomorfizm Currego Howarda dla cześci implikacyjnej logiki intuicjonistycznej i typowanego rachunku lambda.
	\item Udowodnij (korzystając z odpowiednego modelu Kripkego), że \( p \vee \neg p \) nie jest tautologią logiki intuicjonistycznej. (Podpowiedź : istnieje model o \( |C| = 2\), dla którego ta formuła nie jest prawdziwa)
\end{enumerate}


\end{document}


